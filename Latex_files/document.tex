\documentclass[11pt,a4paper]{report}
\usepackage[utf8]{inputenc}
\usepackage{amsmath}
\usepackage{amsfonts}
\usepackage{amssymb}
\usepackage{graphicx}
\usepackage[left=2cm,right=2cm,top=2cm,bottom=2cm]{geometry}
\author{Avdhut Kamble 142203010}
\title{Assignment 4}
\begin{document}
	\maketitle
	
	1. Simple Method\\
	
	2. BiBtex Method\\
	
	Moving Average Crossover: After graphing, two 
	moving averages based on separate time periods tend to cross, 
	which is known as a moving-average crossover ~\cite{abc}. A quicker 
	moving average and a slower moving average are used in this 
	indication (or more). The shorter moving average (short-term) ~\cite{pqr}
	can be 5, 10, or 15 days, while the longer-term moving ~\cite{aa,trishna}
	average might be 100, 200, or 250 days. Since it only 
	evaluates prices over a short period of time, a short-term 
	moving average is speedier and more responsive to daily 
	price changes ~\cite{trishna,sachi,rucha}
	
	
	\begin{thebibliography} {}
		
		\bibitem {aa}Trishna Ugale.,Rucha Kardile.,Stock Price Predictions using Crossover SMA,978-1-6654-1703-7/21.
		
		\bibitem{trishna} Trishna Ugale.,Rucha Kardile.,Stock Price Predictions using Crossover SMA,978-1-6654-1703-7/21.
		
		\bibitem{pqr} PQR,Latex ,IEEE
		
		\bibitem{sachi} Sachi Nandan Mohanty.,Rucha Kardile.,Stock Price Predictions using Crossover SMA,978-1-6654-1703-7/21.
		
		\bibitem{rucha} Trishna Ugale.,Sachi Nandan Mohanty.,Stock Price Predictions using Crossover SMA,978-1-6654-1703-7/21.
		
		\bibitem{abc} ABC,Research Paper,2022,IEEE.
		
		
		
	\end{thebibliography} 
	
\end{document}